\documentclass[11pt,twocolumn,twoside]{bopHomework}
\setboolean{shortarticle}{true}
\setboolean{minireview}{true}
\setboolean{displayabstract}{false}
\setboolean{displaycopyright}{false}

\usepackage{lettrine}
\usepackage{tikz}

\title{Commentaire sur <<~\textit{A simple statistical guide for the analysis
  of behaviour when data are constrained due to practical or ethical
  reasons}~>> \cite{garamszegi2016}
}

\author{Xavier Laviron}

\begin{document}

\maketitle


\section{Il était une fois la p-value}


\section{Comparaison entre approches classique et bayésienne}

Pour faire face à ces problèmes, l'approche bayésienne est une alternative
intéressante.
Une simple recherche sur \textit{Web Of Science} montre une nette augmentation
des articles utilisant cette approche en écologie ou en biologie (figure
\ref{fig:bibliométrie}).

\begin{figure}[h]
  \centering{\graphfont\input{img/dyn_bibliometry.tex}}
  \caption{Évolution du nombre de publications traitant d'écologie ou de
    biologie et utilisant une approche bayésienne entre 1993 et 2016 (source :
    \textit{Web Of Science})}
  \label{fig:bibliométrie}
\end{figure}


\section{Dans d'autres domaines}


\section{Un point sur le partage de données}

L'utilisation de l'approche bayésienne est surtout méta-analytique, elle gagne
donc en intérêt si les données des études précédentes sont disponibles.


\begin{footnotesize}
  \bibliography{biblio.bib}
\end{footnotesize}

\end{document}
