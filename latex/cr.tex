\documentclass[11pt,twocolumn,twoside]{bopHomework}
\setboolean{shortarticle}{true}
\setboolean{minireview}{true}
\setboolean{displayabstract}{false}
\setboolean{displaycopyright}{false}

\usepackage{lettrine}

\title{Commentaire sur <<~\textit{A simple statistical guide for the analysis
  of behaviour when data are constrained due to practical or ethical
  reasons}~>> \cite{garamszegi2016}
}

\author{Xavier Laviron}

\begin{document}

\maketitle


\section{Des données et des Hommes}

\lettrine{E}{n écologie}, les chercheurs sont souvent confrontés à d'importants problèmes
concernant les données analysées.
Ces problèmes sont souvent inhérents aux sujets analysés et ne peuvent donc pas
être corrigés.
Un des problèmes fréquents est la limite de la taille de l'échantillon.
Ce problème est d'autant plus important quand il s'agit de travailler sur des
espèces en danger.
En effet leur effectif est souvent limité et les méthodes d'échantillonnage
disponibles doivent respecter de nombreuses contraintes.
Il est d'autant plus important de trouver une solution à ce problème que les
résultats de telles études peuvent mener à l'établissement de politiques de
conservation dont peut dépendre la survie de l'espèce concernée.

Une solution à ces problèmes est l'utilisation d'outils statistques
alternatifs, tels que l'inférence bayésienne.

\section{Comparaison entre approches classique et bayésienne}

\lipsum[1-2]

\section{Dans d'autres domaines}

\lipsum[1-2]

\section{Autre section}

\lipsum[1-3]

\section{Autre section encore}

\lipsum[1-2]

\begin{footnotesize}
  \bibliography{biblio.bib}
\end{footnotesize}

\end{document}
